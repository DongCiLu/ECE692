
\documentclass[a4paper,conference]{IEEEtran}
% Some Computer Society conferences also require the compsoc mode option,
% but others use the standard conference format.
%
% If IEEEtran.cls has not been installed into the LaTeX system files,
% manually specify the path to it like:
% \documentclass[conference]{../sty/IEEEtran}

\usepackage{cite}
\usepackage[pdftex]{graphicx}
\usepackage{amsmath}
\usepackage{algorithm2e}
\usepackage{subfigure}
\usepackage{threeparttable}
\usepackage{color}

\newcommand{\note}[1]{\textbf{\color{blue}#1}}

\begin{document}
%
% paper title
% Titles are generally capitalized except for words such as a, an, and, as,
% at, but, by, for, in, nor, of, on, or, the, to and up, which are usually
% not capitalized unless they are the first or last word of the title.
% Linebreaks \\ can be used within to get better formatting as desired.
% Do not put math or special symbols in the title.
\title{Image Sentiment Analysis\\ with Convolutional Networks}


% author names and affiliations
% use a multiple column layout for up to three different
% affiliations
\author{\IEEEauthorblockN{Zheng Lu, Yunhe Feng}
\IEEEauthorblockA{
Department of Electrical Engineering and Computer Science\\
University of Tennessee, Knoxville\\
Email: \{zlu12, yfeng14\}@vols.utk.edu}
}

% make the title area
\maketitle

% As a general rule, do not put math, special symbols or citations
% in the abstract
\begin{abstract}
Understanding the underlying attitude of people towards a given image automatically is important for many applications. The rapid growth of social media provides new opportunities and challenges to design image sentiment inference systems. Due to the weak relation between low-level visual features and sentiment, recent work focus on constructing mid-level semantic representation that can both be easily detected from the input images and be mapped to the sentiment class. However, ... We propose a novel convolution network based end to end system that can automatically generate the most suitable mid-level features. We show the effectiveness of our approaches by detecting sentiment of image tweets and show significant improvement against baseline approaches.
\end{abstract}

\IEEEpeerreviewmaketitle

\section{Introduction}
\label{introduction}

Recent years, social media platforms have seen a rapid growth of user-generated multimedia contents. Billion of images are shared on multiple social media platforms such as Instagram or Twitter which contribute to a large portion of the shared links. Through image sharing, users are usually also express their emotions and sentiments to strengthen the opinion carried in the content. Understanding users' sentiments provides us reliable signals of people's real-world activities which is very helpful in many applications such as predicting movie box-office revenues \cite{asur2010predicting}, political voting forecasts \cite{o2010tweets}. It can also be used as the building block for other tasks such as the image captioning \cite{vinyals2015show}.

Automatic sentiment analysis recognize a person's position, attitude or opinion on an entity with computer technologies \cite{soleymani2017survey}. Text-based sentiment analysis has been the main concentration in the past. Only recently, sentiment analysis from online social media images has begun to draw more attentions. To simplify the task, previously the sentiment analysis mainly focus on the opinion's polarity, i.e., one's sentiment is classified into categories of positive, neutral and negative. 
However, as pointed out in many recent studies \cite{borth2013large, yuan2013sentribute, chen2014deepsentibank, ahsan2017towards}, it faces the unique challenge of large "affective gap" between the low-level features and the high-level sentiment. 

To overcome this challenge, recent work resort to extract manually designed mid-level representations from low-level features, e.g., visual sentiment ontology (VSO) in \cite{borth2013large}, mid-level attributes in \cite{yuan2013sentribute}. Such approaches usually outperforms methods that inferring sentiment directly from low-level features. Rapid developments in Convolutional Neural Networks (CNNs) \cite{krizhevsky2012imagenet, szegedy2015going, simonyan2014very, he2016deep} push the transformations of computer vision tasks. There are also several efforts to apply CNNs to image sentiment analysis \cite{you2015robust, chen2014deepsentibank, campos2017pixels}. However, recent works on applying convolutional networks still borrow the network architectures from the image classification tasks \cite{you2015robust, chen2014deepsentibank, ahsan2017towards, campos2017pixels}. Such a methodology limits the proposed systems works only on images containing object, person or scene. We propose...
\note{another weakness lies in the generalizability to cover different domains.}
  
\subsection{Contributions}
Our contributions can be summarized as follows:

\begin{itemize}
	\item We propose ...;
	\item We design ...;
	\item Experiments on ....
\end{itemize}

The rest of this paper is organized as follows. In Section~\ref{relatedwork} we show previous works on image sentiment analysis.  
We explain ... in Section~\ref{design}. 
The evaluations of our proposed method are in Section~\ref{evaluation}. 
We conclude our work in Section~\ref{conclusion}.

\section{Related Works}
\label{relatedwork} 

\note{simply write down what did each paper do for now}

The majority of work in image sentiment analysis focus on ... .

\cite{borth2013large} trys to fill in the "affective gap" between the low-level features and the high-level sentiment by a set of mid-level representation called visual sentiment ontology that consist of more than 3,000 Adjective Noun Pairs (ANP) such as "beautiful flower" or "disgusting food". The authors also publish a large-scale dataset called SentiBank that is widely used in later works. They also extends their work into a multilingual settings in one of their later work \cite{jou2015visual}. 

\cite{you2015robust} applies CNN to image sentiment analysis. They propose a method to progressively training CNNs by keep training instances with distinct sentiment scores towards sentiment polars and discard training instances otherwise.

\cite{yuan2013sentribute} adopts a similar methodology as \cite{borth2013large}. The authors also construct a mid-level attributes for better classification except that they choose different scene-based mid-level attributes than \cite{borth2013large}. In addition to that, they also include face detection to enhance the performance on images containing human face.

\note{the challenge of collecting data: not only the image need to be manually labeled, but also each image normally require multiple people to label it as it is very common for people to have different opinion on a same image. In the end, only a portion of the labeled data can be used, for example, images with all 5 annotator agree on the sentiment label.}

Due to the challenge in manually collect labeled data for image sentiment analysis, \cite{wang2015unsupervised} proposes an unsupervised method to facilitate social media images sentiment analysis with textual information associated with each image.

\cite{chen2014deepsentibank} try to apply CNN based on AlexNet \cite{krizhevsky2012imagenet} to automaticall extract features based on ANPs proposed in \cite{borth2013large}.

\cite{ahsan2017towards} studies the sentiment analysis of images of social events. It designs specific mid-level representations of each event class and classify sentiment of each image without the help of texts associated with the image.

\cite{campos2017pixels} studies how to fine-tune AlexNet-styled CNNs to achieve better performance on image sentiment analysis tasks.

Our work falls into the category of ... .

\section{The design}
\label{design}

We propose to solve the ... .

\subsection{Preliminary}
In our problem ...

\subsection{Main design}

\subsection{More details}


\section{Evaluations}
\label{evaluation}

In this section, we show the results of experimental evaluation of ...

\subsection{Datasets}
\label{eval_datasets}

\begin{table}
		\vspace{-0.5cm}
		\caption{Datasets}
		%\vspace{2 mm}
		\label{table:datasets}
		\begin{threeparttable}
			\centering
			\begin{tabular}{l|ccc} \hline
				Dataset & size & mid-level size & label \\ \hline
				SentiBank-Flickr \cite{borth2013large, chen2014deepsentibank, jou2015visual} & ~316,000 & 3244 & auto \\ 
				SentiBank-twitter \cite{borth2013large, chen2014deepsentibank, jou2015visual} & 1269 & N/A & manual-5 \\
				twitter \cite{yuan2013sentribute} & 14,340 & ~800 & manual \\
				Flickr \cite{wang2015unsupervised} & 20,000 & N/A & manual \\
				Bing \cite{ahsan2017towards} & 10,500 & 24 & manual-3 \\ \hline
			\end{tabular}
			\begin{tablenotes}
				\item Datasets with the no. of images and no. of mid-level representations if applicable. We also list how the label is generated, the number following manual method shows how many workers label each image.
			\end{tablenotes}
		\end{threeparttable}
		\vspace{-0.3cm}
\end{table}

We evaluate our algorithm on ...

\subsection{Experiment settings}
\label{eval_system}

We evaluate our algorithms in ... setting

\subsection{Classification Accuracy}
\label{eval_accuracy}

We show the results...

\subsection{More}
\label{eval_more}


We show in this section that ...

\section{Conclusion}
\label{conclusion}

In this paper, we describe a novel method to ...

% use section* for acknowledgment
%\section*{Acknowledgment}


%The authors would like to thank...

\bibliographystyle{abbrv}
\bibliography{reference}


% that's all folks
\end{document}


